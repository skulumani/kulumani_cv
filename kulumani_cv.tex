
%%%%%%%%%%%%%%%%%%%%%%%%%%%% Document Setup %%%%%%%%%%%%%%%%%%%%%%%%%%%%

% Don't like 10pt? Try 11pt or 12pt
\documentclass[10pt]{article}
\RequirePackage[T1]{fontenc}
\input{my_macros}
% LaTeX will typeset using Computer Modern Roman, which a lot of
% non-mathematicians and non-engineers won't like. Also, a few PDF
% viewers may not render CMR very well. Instead, Times New Roman can
% be used. That's what this package does.
\usepackage{times}

% The automated optical recognition software used to digitize resume
% information works best with fonts that do not have serifs. This
% command uses a sans serif font throughout. Uncomment both lines (or at
% least the second) to restore a Roman font (i.e., a font with serifs).
% (NOTE: This requires the times package above)
%\renewcommand{\familydefault}{\sfdefault}

% This is a helpful package that puts math inside length specifications
\usepackage{calc}

% This package helps LaTeX auto-hyphenate hyphenated words if you use
% special hyphens. For example, bio\-/mimicry will properly hyphenate
% ``mimicry'' if necessary.
\usepackage[shortcuts]{extdash}

% Layout: Puts the section titles on left side of page
\reversemarginpar

%
%         PAPER SIZE, PAGE NUMBER, AND DOCUMENT LAYOUT NOTES:
%
% The next \usepackage line changes the layout for CV style section
% headings as marginal notes. It also sets up the paper size as either
% letter or A4. By default, letter was used. If A4 paper is desired,
% comment out the letterpaper lines and uncomment the a4paper lines.
%
% As you can see, the margin widths and section title widths can be
% easily adjusted.
%
% ALSO: Notice that the includefoot option can be commented OUT in order
% to put the PAGE NUMBER *IN* the bottom margin. This will make the
% effective text area larger.
%
% IF YOU WISH TO REMOVE THE ``of LASTPAGE'' next to each page number,
% see the note about the +LP and -LP lines below. Comment out the +LP
% and uncomment the -LP.
%
% IF YOU WISH TO REMOVE PAGE NUMBERS, be sure that the includefoot line
% is uncommented and ALSO uncomment the \pagestyle{empty} a few lines
% below.
%

%% Use these lines for letter-sized paper
\usepackage[paper=letterpaper,
            %includefoot, % Uncomment to put page number above margin
            marginparwidth=1.2in,     % Length of section titles
            marginparsep=.05in,       % Space between titles and text
            margin=1in,               % 1 inch margins
            includemp]{geometry}

%% Use these lines for A4-sized paper
%\usepackage[paper=a4paper,
%            %includefoot, % Uncomment to put page number above margin
%            marginparwidth=30.5mm,    % Length of section titles
%            marginparsep=1.5mm,       % Space between titles and text
%            margin=25mm,              % 25mm margins
%            includemp]{geometry}

%% More layout: Get rid of indenting throughout entire document
\setlength{\parindent}{0in}

% Provides special list environments and macros to create new ones
\usepackage[shortlabels]{enumitem}

\usepackage[backend=biber]{biblatex}
\addbibresource{library.bib}
\ExecuteBibliographyOptions{}
\AtEveryCitekey{%
\clearfield{url}%
}


%% Reference the last page in the page number
%
% NOTE: comment the +LP line and uncomment the -LP line to have page
%       numbers without the ``of ##'' last page reference)
%
% NOTE: uncomment the \pagestyle{empty} line to get rid of all page
%       numbers (make sure includefoot is commented out above)
%
\usepackage{fancyhdr,lastpage}
\pagestyle{fancy}
%\pagestyle{empty}      % Uncomment this to get rid of page numbers
\fancyhf{}\renewcommand{\headrulewidth}{0pt}
\fancyfootoffset{\marginparsep+\marginparwidth}
\newlength{\footpageshift}
\setlength{\footpageshift}
          {0.5\textwidth+0.5\marginparsep+0.5\marginparwidth-2in}
\lfoot{\hspace{\footpageshift}%
       \parbox{4in}{\, \hfill %
                    \arabic{page} of \protect\pageref*{LastPage} % +LP
%                    \arabic{page}                               % -LP
                    \hfill \,}}

% Finally, give us PDF bookmarks
\usepackage{color,hyperref}
\definecolor{darkblue}{rgb}{0.0,0.0,0.3}
\hypersetup{colorlinks,breaklinks,
            linkcolor=darkblue,urlcolor=darkblue,
            anchorcolor=darkblue,citecolor=darkblue}

%%%%%%%%%%%%%%%%%%%%%%%% End Document Setup %%%%%%%%%%%%%%%%%%%%%%%%%%%%


%%%%%%%%%%%%%%%%%%%%%%%%%%% Helper Commands %%%%%%%%%%%%%%%%%%%%%%%%%%%%

%%% HEADING AT TOP OF CURRICULUM VITAE

% The title (name) with a horizontal rule under it
% (optional argument typesets an object right-justified across from name
%  as well)
%
% Usage: \makeheading{name}
%        OR
%        \makeheading[right_object]{name}
%
% Place at top of document. It should be the first thing.
% If ``right_object'' is provided in the square-braced optional
% argument, it will be right justified on the same line as ``name'' at
% the top of the CV. For example:
%
%       \makeheading[\emph{Curriculum vitae}]{Your Name}
%
% will put an emphasized ``Curriculum vitae'' at the top of the document
% as a title. Likewise, a picture could be included:
%
%   \makeheading[{\includegraphics[height=1.5in]{my_picture}}]{Your Name}
%
% the picture will be flush right across from the name. For this example
% to work, make sure the extra set of curly braces is included. Also
% makes ure that \usepackage{graphicx} is somewhere in the preamble.
\newcommand{\makeheading}[2][]%
        {\hspace*{-\marginparsep minus \marginparwidth}%
         \begin{minipage}[t]{\textwidth+\marginparwidth+\marginparsep}%
             {\large \bfseries #2 \hfill #1}\\[-0.15\baselineskip]%
                 \rule{\columnwidth}{1pt}%
         \end{minipage}}

%%% SECTION HEADINGS

% The section headings. Flush left in small caps down pseudo-margin.
%
% Usage: \section{section name}
\renewcommand{\section}[1]{\pagebreak[3]%
    \vspace{1.3\baselineskip}%
    \phantomsection\addcontentsline{toc}{section}{#1}%
    \noindent\llap{\scshape\smash{\parbox[t]{\marginparwidth}{\hyphenpenalty=10000\raggedright #1}}}%
    \vspace{-\baselineskip}\par}

%%% LISTS

% This macro alters a list by removing some of the space that follows the list
% (is used by lists below)
\newcommand*\fixendlist[1]{%
    \expandafter\let\csname preFixEndListend#1\expandafter\endcsname\csname end#1\endcsname
    \expandafter\def\csname end#1\endcsname{\csname preFixEndListend#1\endcsname\vspace{-0.6\baselineskip}}}

% These macros help ensure that items in outer-type lists do not get
% separated from the next line by a page break
% (they are used by lists below)
\let\originalItem\item
\newcommand*\fixouterlist[1]{%
    \expandafter\let\csname preFixOuterList#1\expandafter\endcsname\csname #1\endcsname
    \expandafter\def\csname #1\endcsname{\let\oldItem\item\def\item{\pagebreak[2]\oldItem}\csname preFixOuterList#1\endcsname}
    \expandafter\let\csname preFixOuterListend#1\expandafter\endcsname\csname end#1\endcsname
    \expandafter\def\csname end#1\endcsname{\let\item\oldItem\csname preFixOuterListend#1\endcsname}}
\newcommand*\fixinnerlist[1]{%
    \expandafter\let\csname preFixInnerList#1\expandafter\endcsname\csname #1\endcsname
    \expandafter\def\csname #1\endcsname{\let\oldItem\item\let\item\originalItem\csname preFixInnerList#1\endcsname}
    \expandafter\let\csname preFixInnerListend#1\expandafter\endcsname\csname end#1\endcsname
    \expandafter\def\csname end#1\endcsname{\csname preFixInnerListend#1\endcsname\let\item\oldItem}}

% An itemize-style list with lots of space between items
%
% Usage:
%   \begin{outerlist}
%       \item ...    % (or \item[] for no bullet)
%   \end{outerlist}
\newlist{outerlist}{itemize}{3}
    \setlist[outerlist]{label=\enskip\textbullet,leftmargin=*}
    \fixendlist{outerlist}
    \fixouterlist{outerlist}

% An environment IDENTICAL to outerlist that has better pre-list spacing
% when used as the first thing in a \section
%
% Usage:
%   \begin{lonelist}
%       \item ...    % (or \item[] for no bullet)
%   \end{lonelist}
\newlist{lonelist}{itemize}{3}
    \setlist[lonelist]{label=\enskip\textbullet,leftmargin=*,partopsep=0pt,topsep=0pt}
    \fixendlist{lonelist}
    \fixouterlist{lonelist}

% An itemize-style list with little space between items
%
% Usage:
%   \begin{innerlist}
%       \item ...    % (or \item[] for no bullet)
%   \end{innerlist}
\newlist{innerlist}{itemize}{3}
    \setlist[innerlist]{label=\enskip\textbullet,leftmargin=*,parsep=0pt,itemsep=0pt,topsep=0pt,partopsep=0pt}
    \fixinnerlist{innerlist}

% An environment IDENTICAL to innerlist that has better pre-list spacing
% when used as the first thing in a \section
%
% Usage:
%   \begin{loneinnerlist}
%       \item ...    % (or \item[] for no bullet)
%   \end{loneinnerlist}
\newlist{loneinnerlist}{itemize}{3}
    \setlist[loneinnerlist]{label=\enskip\textbullet,leftmargin=*,parsep=0pt,itemsep=0pt,topsep=0pt,partopsep=0pt}
    \fixendlist{loneinnerlist}
    \fixinnerlist{loneinnerlist}

%%% EXTRA SPACE

% To add some paragraph space between lines.
% This also tells LaTeX to preferably break a page on one of these gaps
% if there is a needed pagebreak nearby.
\newcommand{\blankline}{\quad\pagebreak[3]}
\newcommand{\halfblankline}{\quad\vspace{-0.5\baselineskip}\pagebreak[3]}

%%% FORMATTING MACROS

% Provides a linked \doi{#1} that links doi:#1 to http://dx.doi.org/#1
\usepackage{doi}
% To change the text before the DOI, adjust this command
%\renewcommand\doitext{doi:}

% Provides a linked \url{#1} that doesn't require escape characters
\usepackage{url}

% You can adjust the style \url{} uses here:
% (options are: same, rm, sf, tt; defaults to tt)
\urlstyle{same}

% For \email{ADDRESS}, links ADDRESS to the url mailto:ADDRESS
% (uncomment to typeset the e\-/mail address in typewriter font;
%  otherwise, will be typeset in the \urlstyle above)
%\DeclareUrlCommand\emaillink{\urlstyle{tt}}
\providecommand*\emaillink[1]{\nolinkurl{#1}}
\providecommand*\email[1]{\href{mailto:#1}{\emaillink{#1}}}

\providecommand\BibTeX{{B\kern-.05em{\sc i\kern-.025em b}\kern-.08em \TeX}}
\providecommand\Matlab{\textsc{Matlab}}

% Custom hyphenation rules for words that LaTeX has trouble with
\hyphenation{bio-mim-ic-ry bio-in-spi-ra-tion re-us-a-ble pro-vid-er Media-Wiki}

\newcommand{\CC}{C\nolinebreak\hspace{-.05em}\raisebox{.4ex}{\tiny\bf +}\nolinebreak\hspace{-.10em}\raisebox{.4ex}{\tiny\bf +}}
\def\CC{{C\nolinebreak[4]\hspace{-.05em}\raisebox{.4ex}{\tiny\bf ++}}}
%%%%%%%%%%%%%%%%%%%%%%%% End Helper Commands %%%%%%%%%%%%%%%%%%%%%%%%%%%

%%%%%%%%%%%%%%%%%%%%%%%%% Begin CV Document %%%%%%%%%%%%%%%%%%%%%%%%%%%%

\begin{document}

\makeheading{Shankar~Kulumani}

\section{Contact Information}

% NOTE: Mind where the & separators and \\ breaks are in the following
%       table. Table is one row made up of three parboxes. The left
%       parbox has address info, the middle parbox has a vertical bar,
%       and the right parbox has phone and electronic contact
%       information.
%
% MACROS: \rcollength is the width of the right column of the table
%             (adjust it to your liking; default is 1.85in).
%         \spacewidth is width of area between left and right boxes.
%
\newlength{\rcollength}\setlength{\rcollength}{2.2in}%
\newlength{\spacewidth}\setlength{\spacewidth}{20pt}
%
\begin{tabular}[t]{@{}p{\textwidth-\rcollength-\spacewidth}@{}p{\spacewidth}@{}p{\rcollength}}%

% Address box
\parbox{\textwidth-\rcollength-\spacewidth}{%
2800 Woodley Rd NW\\
Washington, DC  20008  USA}
&
% Uncomment to add a vertical bar in middle of contact information
%{\vrule width 0.5pt}
\parbox[m][2\baselineskip]{\spacewidth}{} &

% Non-snail-mail contact information
\parbox{\rcollength}{%
\textit{Mobile:} 630-336-6257 \\
\textit{E-mail:} \email{skulumani@gwu.edu} \\
\textit{Web: } \url{https://skulumani.github.io}}

\end{tabular}

%%
%% In modern CV's, it seems like ``Objective'' is frowned upon. Instead,
%% incorporate it into a well-constructed cover letter. The ``More
%% information'' can go at the end of the CV, but it should not distract
%% from the section giving references available to contact.
%%
%
% \section{Objective}
%
% Placement in an academic position (i.e., faculty, postdoctoral, or
% research scientist) that allows for advanced research in distributed
% complex adaptive systems (i.e., modeling, analysis, design, and
% verification) with a particular focus on the control of engineered
% agents (e.g., for communications, control, software, electronics, and
% sustainability) and the analysis of biological phenomena (e.g.,
% self-organization, ecological rationality)
% \begin{innerlist}
% \item More information and auxiliary documents can be found at\\\url{http://www.tedpavlic.com/facjobsearch/}
% \end{innerlist}

\section{Research Interests}

\textbf{Dynamics and Control of Aerospace Systems}
Application of geometric mechanics and control to systems evolving on nonlinear manifolds.

%\section{Academic Appointments}
%
%\textbf{Postdoctoral Scholar} \hfill {July 2012 to present}
%\begin{innerlist}
%
%    \item[] \href{http://sols.asu.edu/}{School of Life Sciences},
%            \href{http://www.asu.edu/}{Arizona State University}
%    \begin{innerlist}
%        \item Supervisor: \href{http://www.public.asu.edu/~spratt1}{Professor Stephen C.~Pratt}
%        \item Focus: Behavioral bio-mimicry of animals with emphasis on social insects
%        \item Affiliations:
%            \begin{innerlist}
%                \item \href{http://casi.asu.edu/home}{Complex Adaptive Systems @ ASU}
%                \item \href{http://csdc.asu.edu/}{Center for Social Dynamics and Complexity}
%            \end{innerlist}
%    \end{innerlist}
%
%\end{innerlist}
%
%\halfblankline
%
%\textbf{Postdoctoral Researcher} \hfill {September 2010 to June 2012}
%\begin{innerlist}
%
%    \item[] \href{http://www.cse.ohio-state.edu/}{Department of Computer Science and Engineering},
%            \href{http://www.osu.edu/}{The Ohio State University}
%    \begin{innerlist}
%        \item \href{http://www.nfs.gov/}{National Science Foundation} Cyber-Physical Systems (ENG, \href{http://www.nsf.gov/div/index.jsp?div=eccs}{ECCS})
%        \begin{innerlist}
%            \item[$-$] ``Autonomous Driving in Mixed-Traffic Urban Environments''
%                (grant~\href{http://www.nsf.gov/awardsearch/showAward.do?AwardNumber=0931669}{\#0931669})
%            \item[$-$] Supervisor (co-PI):
%                \href{http://www.cse.ohio-state.edu/~paolo/}%
%                     {Professor Paolo A.~G.~Sivilotti}
%            \item[$-$] PI:
%                \href{http://www.ece.ohio-state.edu/~umit/}%
%                     {Professor \"{U}mit \"{O}zg\"{u}ner}
%        \end{innerlist}
%    \end{innerlist}
%
%\end{innerlist}

\section{Education}

\href{http://www.gwu.edu/}{\textbf{George Washington University}}, Washington, DC
\hfill {Aug~2014-Present}
\begin{lonelist}

\item[] PhD, \href{http://www.mae.seas.gwu.edu/}{Mechanical and Aerospace Engineering}
        \begin{innerlist}
        	\item[] Area of Study: Geometric Mechanics and Control of Aerospace Systems 
			\item[] Advisor: Taeyoung Lee
        \end{innerlist}
\end{lonelist}

\blankline

\href{http://www.purdue.edu/}{\textbf{Purdue University}}, West Lafayette, IN
\hfill {Jan~2011-Dec 2013}
\begin{lonelist}

\item[] M.S.,
        \href{https://engineering.purdue.edu/AAE}
        {Aeronautical and Astronautical Engineering}
        \begin{innerlist}
        	\item[] Area of Study: Astrodynamics, Analytical Mechanics 
        \end{innerlist}
\end{lonelist}

\blankline

\href{http://www.usafa.af.mil/}{\textbf{United States Air Force Academy}}, Colorado Springs, CO 
\hfill {Jun~2005-May~2009}
\begin{lonelist}
\item[] B.S.,
        \href{http://www.usafa.edu/df/dfas/}
             {Astronautical Engineering}

\end{lonelist}

\section{Qualifications and Skills}

Programming: \Matlab, \LaTeX{}, Python, \CC, JPL/SPICE, Satellite Toolkit (STK)  \\
Design: AutoCAD, SolidWorks \\
Hardware: Vicon/PhaseSpace motion capture systems

\section{Professional Experience}

\textbf{George Washington University}, Washington DC
	\begin{lonelist}
		\item[] \textit{Graduate Research/Teaching Assistant} \hfill {Aug~2014-Present}
		\begin{innerlist}
			\item Graduate teaching assistant for Engineering Drawing and Computer graphics course.
			Responsible for teaching the fundamentals of sound engineering design and use of technical design software.
			Developed assignments and examinations to test students ability in applying principles in technical drawing.
			\item Performing research in the Flight Dynamics and Control laboratory with applications in geometric mechanics and control. 
			Developing methods of continuous low thrust orbital transfers by leveraging the nonlinear manifolds of the three body problem.
			\item Applying computational geometric mechanics to develop algorithms which preserve the geometric properties of mechanical systems to obtain accurate numerical results and long term stability.
			\item Designing constrained geometric control for the coupled translation and rotational dynamics of spacecraft.
			This allows for globally defined control systems that track a desired trajectory while simultaneously avoiding obstacles or path constraints.
		\end{innerlist}
		
		\item[] \textit{Space Scholar, AFRL Scholar Program} \hfill {Jun~2015-Jul~2015}
		\begin{innerlist}
			\item Investigated combined translational and rotational control techniques for spacecraft rendezvous and proximity operations in the presence of constraints.
			Developed a geometric nonlinear controller which allows for global attitude tracking.
			\item The control system is developed directly on the nonlinear manifold and defined globally without the need of local parameterizations.
			Attitude constraints are incorporated directly on the nonlinear manifold through the use of barrier functions and allow for excellent convergence properties in the presence of constraints. 
		\end{innerlist}
	\end{lonelist}
	
\blankline

\textbf{United States Air Force, Captain}
\begin{lonelist}

\item[] \textit{Threat Systems Engineer, Missile and Space Intelligence Center \\
				Defense Intelligence Agency~(MSIC/DIA)}
				\hfill {Sep~2014-Present}
		\begin{innerlist}
			\item Responsible for the development of computational tools for the analysis of rocket and missile systems.
			Developed software to accurately model the aerodynamic forces and effects on rocket systems through all phases of flight.
			Accurate modeling of the aerodynamics allows for improved simulation accuracy and performance predictions.
			\item Developed software to accurately interface operational sensor systems.
			Implemented an extensive \Matlab / Python library to simulate and analyze missile systems and sensors. 
		\end{innerlist}
		
\item[] \textit{Lead Test Engineer, Guidance, Navigation, \& Control Group \\
	Air Force Research Laboratory (AFRL/RVSVC)}%
    \hfill {Aug~2011-Sep~2014}
    \begin{innerlist}
        \item Directed a 6 member team in developing orbit determination software and designing an observation campaign for the ANGELS flight experiment.
        An Unscented Kalman Filter~(UKF) was developed to verify and validate the performance of a GPS receiver in geostationary orbit.
        Additionally, a ground based collection strategy, incorporating AFSSN sensors, was developed and tested on operational satellites.
        The AFRL ANGELS vehicle launched in 2014 to advance autonomous rendezvous and proximity operation control algorithms.
        The UKF and ground based observation campaign was critical for accurate navigation and allowed for more aggressive maneuvers. 
		\item Designed custom astrodynamics force model for the ANGLES flight experiment.
		Incorporated the effects of solar radiation pressure, attitude control actuators and sensors, Earth gravitational models, and thruster uncertainties to predict the expected performance of the spacecraft.
		Analysis predicted a coupling between rotational and translational motion during angular momentum desaturation burns.
		A orientation profile was developed which minimized the accumulation of undesired angular momentum and allowed for extended quiescent periods in support of orbit determination activities.
        \item Led \$5M lab development program to develop an in situ attitude dynamics and control simulator.  
        Responsible for procurement, design, and integration of the largest spherical air-bearing platform in the world.
        Developed software to interface with inertial measurement units, motion capture systems, and control hardware to allow for hardware in the loop system testing.
        The facility is currently being used to validate future flight attitude control actuators and algorithms. 
        \item Developed a method for space based geolocation via time difference of arrival signals.
        Implemented geometric techniques to develop a closed form analytical solution to locate a noncooperative electromagnetic signal based on the time of arrival to a formation of satellites.
        \item Led the development of a satellite relative motion simulator using embedded ground based robotic systems.
        Designed the software to allow for accurate scaling and simulation of spacecraft motion.
        Directly implemented a motion capture system on embedded hardware to allow for accurate positioning and control. 
        %\item Managed \$4M DARPA space situational awareness contract by leading diverse team of academia, industry, and government in an effort to develop integrated orbit determination software
    \end{innerlist}

\item[] \textit{Deputy Space Vehicles Lead, Responsive Space Squadron\\}
	\textit{Space Development and Test Directorate (SDTD/SDDR)}%
    \hfill {May~2009-Aug~2011}
    
    \begin{innerlist}
        \item Responsible for development, integration, test, \& launch of ORS-1 (Operationally Responsive Space) satellite
         ORS-1 was the first operational satellite developed under the ORS office and supports US Central Command Battlespace Awareness.
         An accelerated development cycle allowed for the integration and launch in under 2 years.
        \item Extensive experience with technical management of diverse contractor/government teams leading to successful ORS-1 launch and orbit operation.
        Managed a team of 5 to ensure the correct integration and testing of the space vehicle.
        Ensured mission requirements were being met during the integration period. 
        \item Resolved a \$600K satellite flight sensor failure.
        Directly managed the repair analysis team to prevent launch delays and ensure the system capability.
        Monitored the hardware repair process and verified they were completed correctly.
		\item Firsthand experience monitoring 100+ days of integration and testing of ORS-1.
		Sole space vehicles lead during launch site operations at Wallops Island, VA.
		Ensured correct procedures and standards leading to on-time launch on 29 June 2011
		\item Assessed and served as on-site government inspector of 200+ satellite test plans.
		 Verified technical analysis and testing procedures or all flight hardware of ORS-1. 
		 Test plans were critical in ensuring correct performance of both ground and space hardware and critical to mission success.
    \end{innerlist}

\end{lonelist} % AF work

\section{Publications} 
\begin{lonelist}	
    \item[] \textbf{Journal Publications}
    \begin{innerlist}
        \item \fullcite{kulumani2016e}
        \item \fullcite{kulumani2016a}
    \end{innerlist}
    \item[] \textbf{Conference Publications}
    \begin{innerlist}
        \item \fullcite{kulumani2016d}
        \item \fullcite{kulumani2016c}
        \item \fullcite{kulumani2016}
        \item \fullcite{kulumani2015}
        \item \fullcite{kulumani2013} 
    \end{innerlist}
\end{lonelist}
\section{Professional Memberships}

Institute of Electrical and Electronics Engineers~(IEEE), Member \hfill {2016-present}

American Astronautical Society~(AAS), Member \hfill {2015-present}

American Institute of Aeronautics and Astronautics~(AIAA), Member \hfill {2012-present}

Sigma Gamma Tau, Member \hfill  {2008-present}

%\halfblankline use between items

%\halfblankline
%
%Technical Training
%\begin{innerlist}
%\item First aid training including Self Aid Buddy Care~(SABC), CPR Heartsaver
%\end{innerlist}

\section{Professional Service}
Reviewer for \href{http://www.journals.elsevier.com/advances-in-space-research/}{Advances in Space Research}

\section{Awards} % insert military medals here as well
George Washington University
\begin{innerlist}
    \item Graduate Research Fellowship, George Washington University \hfill {2014-2016}
	\item Best session presentation award, American Control Conference (ACC) \hfill {2016}
	\item Student Travel Award, American Control Conference (ACC) \hfill {2016}
	\item Hetherington Familiy Scholarship, George Washington University \hfill {2016}
	\item{1st Place Experimental Research Award, Research and Development Showcase\\ George Washington University} \hfill {2016}
	\item {Most Innovative/Creative Project Award, 5th Annual Student Competition \\ 
	Society of Satellite Professionals International~(SSPI) } \hfill {2015}  
\end{innerlist}

\halfblankline

Responsive Space Squadron
\begin{innerlist}
	\item Rotary National Award for Space Achievement Foundation Stellar Award nomination  for successful ORS-1 mission accomplishments \hfill {2011} 
	\item ORS-1 named by C4ISR Journal as one of the top 25 most important intelligence, surveillance and reconnaissance concepts of the year \hfill {2011}
\end{innerlist}

\halfblankline

United States Air Force Academy
\begin{innerlist}
\item Commandant/Dean pin for high military/academic performance \hfill {2005-2009}
\item Top Academic Performer - Astrodynamics 321 \hfill {2007}
\end{innerlist}

\section{Security Clearance}

Cleared for Top Secret information and granted access to sensitive compartmented information based on a single scope background investigation completed on 21 June 2010.

Please contact SSO AF Reserve Command for additional clearance information.

% \section{Citizenship}
%
% USA

%\section{References Available to Contact}
%
%\href
%{http://www.public.asu.edu/~spratt1/}
%{\textbf{Dr.~Stephen C.~Pratt}}
%(e\-/mail:~\href{mailto:stephen.pratt@asu.edu}{stephen.pratt@asu.edu}; phone:~+1-480-727-9425)
%%
%\begin{innerlist}
%    \item Associate Professor,
%        \href{http://sols.asu.edu/}{School of Life Sciences},
%        \href{http://www.asu.edu/}{Arizona State University}
%
%    \item[$\diamond$] School of Life Sciences, PO Box 874501, Tempe, AZ
%        85287-4501
%
%    \item[$\star$] \emph{Dr.~Pratt is my current postdoctoral supervisor.}
%\end{innerlist}


% The ``More Info'' section may not be necessary; make sure it's short
% so it doesn't prevent people from seeing references available to
% contact.
%\section{More Information}
%
%More information and auxiliary documents can be found at\\%
%\url{http://www.tedpavlic.com/facjobsearch/}.
%

\end{document}

%%%%%%%%%%%%%%%%%%%%%%%%%% End CV Document %%%%%%%%%%%%%%%%%%%%%%%%%%%%%

%----------------------------------------------------------------------%
