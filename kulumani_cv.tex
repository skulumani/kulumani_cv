
%%%%%%%%%%%%%%%%%%%%%%%%%%%% Document Setup %%%%%%%%%%%%%%%%%%%%%%%%%%%%

% Don't like 10pt? Try 11pt or 12pt
\documentclass[10pt]{article}
\RequirePackage[T1]{fontenc}

% LaTeX will typeset using Computer Modern Roman, which a lot of
% non-mathematicians and non-engineers won't like. Also, a few PDF
% viewers may not render CMR very well. Instead, Times New Roman can
% be used. That's what this package does.
\usepackage{times}

% The automated optical recognition software used to digitize resume
% information works best with fonts that do not have serifs. This
% command uses a sans serif font throughout. Uncomment both lines (or at
% least the second) to restore a Roman font (i.e., a font with serifs).
% (NOTE: This requires the times package above)
%\renewcommand{\familydefault}{\sfdefault}

% This is a helpful package that puts math inside length specifications
\usepackage{calc}

% This package helps LaTeX auto-hyphenate hyphenated words if you use
% special hyphens. For example, bio\-/mimicry will properly hyphenate
% ``mimicry'' if necessary.
\usepackage[shortcuts]{extdash}

% Layout: Puts the section titles on left side of page
\reversemarginpar

%
%         PAPER SIZE, PAGE NUMBER, AND DOCUMENT LAYOUT NOTES:
%
% The next \usepackage line changes the layout for CV style section
% headings as marginal notes. It also sets up the paper size as either
% letter or A4. By default, letter was used. If A4 paper is desired,
% comment out the letterpaper lines and uncomment the a4paper lines.
%
% As you can see, the margin widths and section title widths can be
% easily adjusted.
%
% ALSO: Notice that the includefoot option can be commented OUT in order
% to put the PAGE NUMBER *IN* the bottom margin. This will make the
% effective text area larger.
%
% IF YOU WISH TO REMOVE THE ``of LASTPAGE'' next to each page number,
% see the note about the +LP and -LP lines below. Comment out the +LP
% and uncomment the -LP.
%
% IF YOU WISH TO REMOVE PAGE NUMBERS, be sure that the includefoot line
% is uncommented and ALSO uncomment the \pagestyle{empty} a few lines
% below.
%

%% Use these lines for letter-sized paper
\usepackage[paper=letterpaper,
            %includefoot, % Uncomment to put page number above margin
            marginparwidth=1.2in,     % Length of section titles
            marginparsep=.05in,       % Space between titles and text
            margin=1in,               % 1 inch margins
            includemp]{geometry}

%% Use these lines for A4-sized paper
%\usepackage[paper=a4paper,
%            %includefoot, % Uncomment to put page number above margin
%            marginparwidth=30.5mm,    % Length of section titles
%            marginparsep=1.5mm,       % Space between titles and text
%            margin=25mm,              % 25mm margins
%            includemp]{geometry}

%% More layout: Get rid of indenting throughout entire document
\setlength{\parindent}{0in}

% Provides special list environments and macros to create new ones
\usepackage[shortlabels]{enumitem}

% Simpler bibsections for CV sections
% (thanks to natbib for inspiration)
%
% * For lists of references with hanging indents and no numbers:
%
%   \begin{bibsection}
%       \item ...
%   \end{bibsection}
%
% * For numbered lists of references (with hanging indents):
%
%   \begin{bibenum}
%       \item ...
%   \end{bibenum}
%
%   Note that bibenum numbers continuously throughout. To reset the
%   counter, use
%
%   \restartlist{bibenum}
%
%   at the place where you want the numbering to reset.

\makeatletter
\newlength{\bibhang}
\setlength{\bibhang}{1em}
\newlength{\bibsep}
 {\@listi \global\bibsep\itemsep \global\advance\bibsep by\parsep}
\newlist{bibsection}{itemize}{3}
\setlist[bibsection]{label=,leftmargin=\bibhang,%
        itemindent=-\bibhang,
        itemsep=\bibsep,parsep=\z@,partopsep=0pt,
        topsep=0pt}
\newlist{bibenum}{enumerate}{3}
\setlist[bibenum]{label=[\arabic*],resume,leftmargin={\bibhang+\widthof{[999]}},%
        itemindent=-\bibhang,
        itemsep=\bibsep,parsep=\z@,partopsep=0pt,
        topsep=0pt}
\let\oldendbibenum\endbibenum
\def\endbibenum{\oldendbibenum\vspace{-.6\baselineskip}}
\let\oldendbibsection\endbibsection
\def\endbibsection{\oldendbibsection\vspace{-.6\baselineskip}}
\makeatother

%% Reference the last page in the page number
%
% NOTE: comment the +LP line and uncomment the -LP line to have page
%       numbers without the ``of ##'' last page reference)
%
% NOTE: uncomment the \pagestyle{empty} line to get rid of all page
%       numbers (make sure includefoot is commented out above)
%
\usepackage{fancyhdr,lastpage}
\pagestyle{fancy}
%\pagestyle{empty}      % Uncomment this to get rid of page numbers
\fancyhf{}\renewcommand{\headrulewidth}{0pt}
\fancyfootoffset{\marginparsep+\marginparwidth}
\newlength{\footpageshift}
\setlength{\footpageshift}
          {0.5\textwidth+0.5\marginparsep+0.5\marginparwidth-2in}
\lfoot{\hspace{\footpageshift}%
       \parbox{4in}{\, \hfill %
                    \arabic{page} of \protect\pageref*{LastPage} % +LP
%                    \arabic{page}                               % -LP
                    \hfill \,}}

% Finally, give us PDF bookmarks
\usepackage{color,hyperref}
\definecolor{darkblue}{rgb}{0.0,0.0,0.3}
\hypersetup{colorlinks,breaklinks,
            linkcolor=darkblue,urlcolor=darkblue,
            anchorcolor=darkblue,citecolor=darkblue}

%%%%%%%%%%%%%%%%%%%%%%%% End Document Setup %%%%%%%%%%%%%%%%%%%%%%%%%%%%


%%%%%%%%%%%%%%%%%%%%%%%%%%% Helper Commands %%%%%%%%%%%%%%%%%%%%%%%%%%%%

%%% HEADING AT TOP OF CURRICULUM VITAE

% The title (name) with a horizontal rule under it
% (optional argument typesets an object right-justified across from name
%  as well)
%
% Usage: \makeheading{name}
%        OR
%        \makeheading[right_object]{name}
%
% Place at top of document. It should be the first thing.
% If ``right_object'' is provided in the square-braced optional
% argument, it will be right justified on the same line as ``name'' at
% the top of the CV. For example:
%
%       \makeheading[\emph{Curriculum vitae}]{Your Name}
%
% will put an emphasized ``Curriculum vitae'' at the top of the document
% as a title. Likewise, a picture could be included:
%
%   \makeheading[{\includegraphics[height=1.5in]{my_picture}}]{Your Name}
%
% the picture will be flush right across from the name. For this example
% to work, make sure the extra set of curly braces is included. Also
% makes ure that \usepackage{graphicx} is somewhere in the preamble.
\newcommand{\makeheading}[2][]%
        {\hspace*{-\marginparsep minus \marginparwidth}%
         \begin{minipage}[t]{\textwidth+\marginparwidth+\marginparsep}%
             {\large \bfseries #2 \hfill #1}\\[-0.15\baselineskip]%
                 \rule{\columnwidth}{1pt}%
         \end{minipage}}

%%% SECTION HEADINGS

% The section headings. Flush left in small caps down pseudo-margin.
%
% Usage: \section{section name}
\renewcommand{\section}[1]{\pagebreak[3]%
    \vspace{1.3\baselineskip}%
    \phantomsection\addcontentsline{toc}{section}{#1}%
    \noindent\llap{\scshape\smash{\parbox[t]{\marginparwidth}{\hyphenpenalty=10000\raggedright #1}}}%
    \vspace{-\baselineskip}\par}

%%% LISTS

% This macro alters a list by removing some of the space that follows the list
% (is used by lists below)
\newcommand*\fixendlist[1]{%
    \expandafter\let\csname preFixEndListend#1\expandafter\endcsname\csname end#1\endcsname
    \expandafter\def\csname end#1\endcsname{\csname preFixEndListend#1\endcsname\vspace{-0.6\baselineskip}}}

% These macros help ensure that items in outer-type lists do not get
% separated from the next line by a page break
% (they are used by lists below)
\let\originalItem\item
\newcommand*\fixouterlist[1]{%
    \expandafter\let\csname preFixOuterList#1\expandafter\endcsname\csname #1\endcsname
    \expandafter\def\csname #1\endcsname{\let\oldItem\item\def\item{\pagebreak[2]\oldItem}\csname preFixOuterList#1\endcsname}
    \expandafter\let\csname preFixOuterListend#1\expandafter\endcsname\csname end#1\endcsname
    \expandafter\def\csname end#1\endcsname{\let\item\oldItem\csname preFixOuterListend#1\endcsname}}
\newcommand*\fixinnerlist[1]{%
    \expandafter\let\csname preFixInnerList#1\expandafter\endcsname\csname #1\endcsname
    \expandafter\def\csname #1\endcsname{\let\oldItem\item\let\item\originalItem\csname preFixInnerList#1\endcsname}
    \expandafter\let\csname preFixInnerListend#1\expandafter\endcsname\csname end#1\endcsname
    \expandafter\def\csname end#1\endcsname{\csname preFixInnerListend#1\endcsname\let\item\oldItem}}

% An itemize-style list with lots of space between items
%
% Usage:
%   \begin{outerlist}
%       \item ...    % (or \item[] for no bullet)
%   \end{outerlist}
\newlist{outerlist}{itemize}{3}
    \setlist[outerlist]{label=\enskip\textbullet,leftmargin=*}
    \fixendlist{outerlist}
    \fixouterlist{outerlist}

% An environment IDENTICAL to outerlist that has better pre-list spacing
% when used as the first thing in a \section
%
% Usage:
%   \begin{lonelist}
%       \item ...    % (or \item[] for no bullet)
%   \end{lonelist}
\newlist{lonelist}{itemize}{3}
    \setlist[lonelist]{label=\enskip\textbullet,leftmargin=*,partopsep=0pt,topsep=0pt}
    \fixendlist{lonelist}
    \fixouterlist{lonelist}

% An itemize-style list with little space between items
%
% Usage:
%   \begin{innerlist}
%       \item ...    % (or \item[] for no bullet)
%   \end{innerlist}
\newlist{innerlist}{itemize}{3}
    \setlist[innerlist]{label=\enskip\textbullet,leftmargin=*,parsep=0pt,itemsep=0pt,topsep=0pt,partopsep=0pt}
    \fixinnerlist{innerlist}

% An environment IDENTICAL to innerlist that has better pre-list spacing
% when used as the first thing in a \section
%
% Usage:
%   \begin{loneinnerlist}
%       \item ...    % (or \item[] for no bullet)
%   \end{loneinnerlist}
\newlist{loneinnerlist}{itemize}{3}
    \setlist[loneinnerlist]{label=\enskip\textbullet,leftmargin=*,parsep=0pt,itemsep=0pt,topsep=0pt,partopsep=0pt}
    \fixendlist{loneinnerlist}
    \fixinnerlist{loneinnerlist}

%%% EXTRA SPACE

% To add some paragraph space between lines.
% This also tells LaTeX to preferably break a page on one of these gaps
% if there is a needed pagebreak nearby.
\newcommand{\blankline}{\quad\pagebreak[3]}
\newcommand{\halfblankline}{\quad\vspace{-0.5\baselineskip}\pagebreak[3]}

%%% FORMATTING MACROS

% Provides a linked \doi{#1} that links doi:#1 to http://dx.doi.org/#1
\usepackage{doi}
% To change the text before the DOI, adjust this command
%\renewcommand\doitext{doi:}

% Provides a linked \url{#1} that doesn't require escape characters
\usepackage{url}

% You can adjust the style \url{} uses here:
% (options are: same, rm, sf, tt; defaults to tt)
\urlstyle{same}

% For \email{ADDRESS}, links ADDRESS to the url mailto:ADDRESS
% (uncomment to typeset the e\-/mail address in typewriter font;
%  otherwise, will be typeset in the \urlstyle above)
%\DeclareUrlCommand\emaillink{\urlstyle{tt}}
\providecommand*\emaillink[1]{\nolinkurl{#1}}
\providecommand*\email[1]{\href{mailto:#1}{\emaillink{#1}}}

\providecommand\BibTeX{{B\kern-.05em{\sc i\kern-.025em b}\kern-.08em \TeX}}
\providecommand\Matlab{\textsc{Matlab}}

% Custom hyphenation rules for words that LaTeX has trouble with
\hyphenation{bio-mim-ic-ry bio-in-spi-ra-tion re-us-a-ble pro-vid-er Media-Wiki}

%%%%%%%%%%%%%%%%%%%%%%%% End Helper Commands %%%%%%%%%%%%%%%%%%%%%%%%%%%

%%%%%%%%%%%%%%%%%%%%%%%%% Begin CV Document %%%%%%%%%%%%%%%%%%%%%%%%%%%%

\begin{document}
\makeheading{Capt~Shankar~Kulumani}

\section{Contact Information}

% NOTE: Mind where the & separators and \\ breaks are in the following
%       table. Table is one row made up of three parboxes. The left
%       parbox has address info, the middle parbox has a vertical bar,
%       and the right parbox has phone and electronic contact
%       information.
%
% MACROS: \rcollength is the width of the right column of the table
%             (adjust it to your liking; default is 1.85in).
%         \spacewidth is width of area between left and right boxes.
%
\newlength{\rcollength}\setlength{\rcollength}{2.2in}%
\newlength{\spacewidth}\setlength{\spacewidth}{20pt}
%
\begin{tabular}[t]{@{}p{\textwidth-\rcollength-\spacewidth}@{}p{\spacewidth}@{}p{\rcollength}}%

% Address box
\parbox{\textwidth-\rcollength-\spacewidth}{%
1818 Anderson PL SE\\
Albuquerque, NM  87108  USA}
&
% Uncomment to add a vertical bar in middle of contact information
%{\vrule width 0.5pt}
\parbox[m][2\baselineskip]{\spacewidth}{} &

% Non-snail-mail contact information
\parbox{\rcollength}{%
\textit{Mobile:} 630-336-6257 \\
\textit{E-mail:} \email{shankar.kulumani@gmail.com}}

\end{tabular}

%%
%% In modern CV's, it seems like ``Objective'' is frowned upon. Instead,
%% incorporate it into a well-constructed cover letter. The ``More
%% information'' can go at the end of the CV, but it should not distract
%% from the section giving references available to contact.
%%
%
% \section{Objective}
%
% Placement in an academic position (i.e., faculty, postdoctoral, or
% research scientist) that allows for advanced research in distributed
% complex adaptive systems (i.e., modeling, analysis, design, and
% verification) with a particular focus on the control of engineered
% agents (e.g., for communications, control, software, electronics, and
% sustainability) and the analysis of biological phenomena (e.g.,
% self-organization, ecological rationality)
% \begin{innerlist}
% \item More information and auxiliary documents can be found at\\\url{http://www.tedpavlic.com/facjobsearch/}
% \end{innerlist}

\section{Research Interests}

\textbf{Astronautical Engineering with applications in control systems theory:}
Focus on spacecraft attitude dynamics and control, estimation and orbit determination

%\section{Academic Appointments}
%
%\textbf{Postdoctoral Scholar} \hfill {July 2012 to present}
%\begin{innerlist}
%
%    \item[] \href{http://sols.asu.edu/}{School of Life Sciences},
%            \href{http://www.asu.edu/}{Arizona State University}
%    \begin{innerlist}
%        \item Supervisor: \href{http://www.public.asu.edu/~spratt1}{Professor Stephen C.~Pratt}
%        \item Focus: Behavioral bio-mimicry of animals with emphasis on social insects
%        \item Affiliations:
%            \begin{innerlist}
%                \item \href{http://casi.asu.edu/home}{Complex Adaptive Systems @ ASU}
%                \item \href{http://csdc.asu.edu/}{Center for Social Dynamics and Complexity}
%            \end{innerlist}
%    \end{innerlist}
%
%\end{innerlist}
%
%\halfblankline
%
%\textbf{Postdoctoral Researcher} \hfill {September 2010 to June 2012}
%\begin{innerlist}
%
%    \item[] \href{http://www.cse.ohio-state.edu/}{Department of Computer Science and Engineering},
%            \href{http://www.osu.edu/}{The Ohio State University}
%    \begin{innerlist}
%        \item \href{http://www.nfs.gov/}{National Science Foundation} Cyber-Physical Systems (ENG, \href{http://www.nsf.gov/div/index.jsp?div=eccs}{ECCS})
%        \begin{innerlist}
%            \item[$-$] ``Autonomous Driving in Mixed-Traffic Urban Environments''
%                (grant~\href{http://www.nsf.gov/awardsearch/showAward.do?AwardNumber=0931669}{\#0931669})
%            \item[$-$] Supervisor (co-PI):
%                \href{http://www.cse.ohio-state.edu/~paolo/}%
%                     {Professor Paolo A.~G.~Sivilotti}
%            \item[$-$] PI:
%                \href{http://www.ece.ohio-state.edu/~umit/}%
%                     {Professor \"{U}mit \"{O}zg\"{u}ner}
%        \end{innerlist}
%    \end{innerlist}
%
%\end{innerlist}

\section{Education}

\href{http://www.purdue.edu/}{\textbf{Purdue University}},
West Lafayette, IN
\hfill \textbf{January 2011 to December 2013}
\begin{outerlist}

\item[] M.S.,
        \href{https://engineering.purdue.edu/AAE}
        {Aeronautics and Astronautics Engineering}
        \begin{innerlist}
        	\item Overall GPA: 3.66/4.00
        	\item Area of Study: Spacecraft Dynamics and Control 
        \end{innerlist}

\end{outerlist}

\halfblankline

\href{http://www.usafa.af.mil/}{\textbf{United States Air Force Academy}},
Colorado Springs, CO 
\hfill \textbf{June 2005 to May 2009}
\begin{outerlist}
\item[] B.S.,
        \href{http://www.usafa.edu/df/dfas/}
             {Astronautical Engineering}
        \begin{innerlist}
			\item Overall GPA: 3.35/4.00
        \end{innerlist}

\end{outerlist}

\section{Professional Experience}

\textbf{United States Air Force}, Kirtland AFB, NM
\begin{lonelist}

\item[] \textit{Lead Test Engineer, Air Force Research Laboratory}%
    \hfill \textbf{August~2011 to July~2014}
    \begin{innerlist}
        \item Created orbit determination software for geo-stationary GPS receiver validation
	\item Designed astrodynamics force model for AFRL satellite science experiment
        \item Developed attitude control simulations for CMG test-bed known as Attitude Control System Proving~(ACSPG) ground
        \item Developed ground transmitter geolocation via satellite time difference of arrival algorithm
        \item Led incorporation of satellite relative motion dynamics, guidance and control for simulation on embedded ground based robotic system
        \item Implemented miniature inertial measurement (IMU) sensors for attitude control experiments
        \item Managed space situational awareness  software development by leading diverse team of academia, industry, and government in an effort to develop integrated orbit determination software
    \end{innerlist}

\item[] \textit{Deputy Space Vehicles Lead, Responsive Space Squadron}%
    \hfill \textbf{May 2009 to August 2011}
    \begin{innerlist}
        \item Responsible for development, integration, test, \& launch of ORS-1 satellite
        \item Extensive experience with technical management of diverse contractor/government teams
        \item Resolved \$600K satellite hardware issues and prevented ORS-1 launch delays
	\item First hand experience monitoring 100+ days of integration and testing of ORS-1 satellite
	\item Assessed 200+ satellite test plans leading to successful test campaign and mitigated possible launch delays
    \end{innerlist}

\end{lonelist}

\section{Professional Memberships}

American Institute of Aeronautics and Astronautics~(AIAA), Member,
2012--present

\halfblankline

Sigma Gamma Tau, Member, 2008--present

%\halfblankline use between items

\section{Qualifications and Skills}

\href{http://www.mathworks.com/products/matlab/}{\Matlab/Simulink} skill set:
%
\begin{innerlist}
    \item  Dynamical system simulation, astrodynamics applications, Linear algebra, Monte Carlo analysis, Optimization,
        GUI development, statistics, estimation, data processing, visualization
\end{innerlist}

\halfblankline

Design Software:
\begin{innerlist}
	\item Solidworks, AutoCAD
\end{innerlist}

\halfblankline

Computer Programming:
%
\begin{innerlist}
    \item Experience with C, C$+$$+$, UNIX shell scripting, DVCS (Git)
\end{innerlist}

\halfblankline

Desktop Editing and Productivity Software:
%
\begin{innerlist}
    \item \TeX{} (\LaTeX{}, \BibTeX{}, PSTricks),
    \item Microsoft Office, OpenOffice.org, LibreOffice, Google Docs
    \item GIMP, InkScape
\end{innerlist}

\halfblankline

Operating Systems:
%
\begin{innerlist}
    \item Microsoft Windows family, Apple OS X, Linux/UNIX
\end{innerlist}

\halfblankline

Hardware Systems:
\begin{innerlist}
	\item PhaseSpace motion capture system
	\item VectoNav Inertial Measurement Unit
	\item Embedded robotic systems
\end{innerlist}

%\halfblankline
%
%Technical Training
%\begin{innerlist}
%\item First aid training including Self Aid Buddy Care~(SABC), CPR Heartsaver
%\end{innerlist}

\section{Expertise}

Control Theory and Engineering:
%
\begin{innerlist}
    \item Linear and Nonlinear Systems Theory, Feedback, Optimization, Digital Control
\end{innerlist}

\halfblankline

Communications and Signal Processing:
%
\begin{innerlist}
    \item Probability, Random Variables, Stochastic Processes, Estimation, Statistical Inference
\end{innerlist}

\halfblankline

Astronautical Engineering:
%
\begin{innerlist}
    \item Astrodynamics, Orbit Determination, Attitude Dynamics, Analytical dynamics, Rocket Propulsion  
\end{innerlist}

\section{Awards}

United States Air Force Academy
\begin{innerlist}
\item Awarded Commandant/Dean pin 8 consecutive semesters for high military/academic performance (2005-2009)
\item Top Academic Performer - Astrodynamics 321 (2007)
\end{innerlist}

\section{Security Clearance}

Department of Defense Top Secret SCI (awarded: 2010)

% \section{Citizenship}
%
% USA

%\section{References Available to Contact}
%
%\href
%{http://www.public.asu.edu/~spratt1/}
%{\textbf{Dr.~Stephen C.~Pratt}}
%(e\-/mail:~\href{mailto:stephen.pratt@asu.edu}{stephen.pratt@asu.edu}; phone:~+1-480-727-9425)
%%
%\begin{innerlist}
%    \item Associate Professor,
%        \href{http://sols.asu.edu/}{School of Life Sciences},
%        \href{http://www.asu.edu/}{Arizona State University}
%
%    \item[$\diamond$] School of Life Sciences, PO Box 874501, Tempe, AZ
%        85287-4501
%
%    \item[$\star$] \emph{Dr.~Pratt is my current postdoctoral supervisor.}
%\end{innerlist}


% The ``More Info'' section may not be necessary; make sure it's short
% so it doesn't prevent people from seeing references available to
% contact.
%\section{More Information}
%
%More information and auxiliary documents can be found at\\%
%\url{http://www.tedpavlic.com/facjobsearch/}.
%
\end{document}

%%%%%%%%%%%%%%%%%%%%%%%%%% End CV Document %%%%%%%%%%%%%%%%%%%%%%%%%%%%%

%----------------------------------------------------------------------%
