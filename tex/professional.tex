\subsection*{Professional Experience}
\phantomsection
\addcontentsline{toc}{section}{Professional Experience}

\textbf{Northrop Grumman}

\begin{itemize}
    \item[] \textit{Astrodynamics Systems Engineer} \hfill {Jan~2021-Present}
        \begin{itemize}
            \item Developing high fidelity astrodynamics software in C++ and Python to support flight missions.
            \item Experience in Agile development to translate customer requirements to software design in order to supporting next generation space systems. 
            \item Providing technical guidance in astrodynamics and attitude dynamics.
        \end{itemize}
\end{itemize}
\textbf{Collins Aerospace}

\begin{itemize}
    \item[] \textit{Guidance, Navigation, and Control Systems Engineer}
                    \hfill {Aug~2018-Dec~2020} 
        \begin{itemize}
            \item Leading five member interdisciplinary team of Engineers supporting an innovative solution for pilot helmet mounted display system. 
                Performing research and development in computer vision and nonlinear estimation to provide an optical aided solution to provide a high integrity estimate of the pose of a pilots heads. 
                Providing techinical guidance and leadership to direct hardware development and customer support.
            \item Performing independent research implementing pupil detection and gaze tracking for head mounted display systems. 
                Developing software tools and algorithms in C++ which implements open source image processing and optimization algorithms.
            \item Implementing aided inertial navigation and sensor fusion algorithms for military and commercial customers. 
                Developing Extended Kalman Filter based inertial navigation algorithms for the \href{https://www.collinsaerospace.com/what-we-do/Military-And-Defense/Navigation/Ground-Products/NavHub-100-navigation-system}{NavHub 100} navigation system which enable high assurance accurate navigation solution in challenging GPS-denied enviornments.
                Through this work, Collins Aerospace was selected by the US Army to provide the next generation of \href{https://www.collinsaerospace.com/en/newsroom/News/2019/10/collins-next-gen-nav-system-selected-us-army-enable-critical-assured-pnt-contested-enviornments}{Mounted Assured Positioning, Navigation, and Timing System} (MAPS).
            \item Conducting independent research in the fields of relative navigation, nonlinear estimation, sensor fusion, and astronomical applications.
                Documenting results in written reports, patent applications, and published documents for internal and external readers. 
            \item Designing and executing test procedures for ensuring avionics systems meet government standards.
                Experience in the translation of requirements to effective software test procedures.
            \item Responsible for the software maintenance of aided inertial navigation and control algorithm software libraries. 
            \item Providing systems support and technical guidance to customers utilizing software algorithms.
        \end{itemize}
\end{itemize}

\textbf{United States Air Force, Captain}
\begin{itemize}

\item[] \textit{Threat Systems Engineer, Missile and Space Intelligence Center \\
                Defense Intelligence Agency~(MSIC/DIA)}
                \hfill {Sep~2014-Sep~2018}
\begin{itemize}
    \item Supported Sensor Week training exercise at Eglin Air Force Base, FL.
    % A wide variety of air and ground vehicles are tested against realistic sensor systems. 
    Directly supported the implementation and deployment of several radar tracking systems and enabled effective air defense against operational flight missions.
    Developed software tools in Python to analyze and display geospatial data from various sensor platforms.
    \item Responsible for the development of computational tools in C++ for the analysis of rocket and missile systems.
    % Developed software to accurately model the aerodynamic forces and effects on rocket systems through all phases of flight.
    % Accurate modeling of the aerodynamics allows for improved simulation accuracy and performance predictions.
    \item Developed software in Python to interface with operational optical and radar sensor.
    Implemented an extensive Python library to simulate and analyze missile systems, flight, and measurements. 
    \item Contributed to Graphical User Interface (GUI) development in C++ using \href{https://www.qt.io/ide/}{Qt Creator}
    GUI serves as graphical software tool used to interface with a variety of missile and aerodynamic simulation tools.
\end{itemize}
        
\item[] \textit{Lead Test Engineer, Guidance, Navigation, \& Control Group \\
    Air Force Research Laboratory (AFRL/RVSVC)}%
    \hfill {Aug~2011-Sep~2014}
\begin{itemize}
    \item Directed a 6 member team in developing orbit determination software and designing an observation campaign for the ANGELS flight experiment.
    % An Unscented Kalman Filter~(UKF) was developed to verify and validate the performance of a GPS receiver in geostationary orbit.
    % Additionally, a ground based collection strategy, incorporating AFSSN sensors, was developed and tested on operational satellites.
    % The AFRL ANGELS vehicle launched in 2014 to advance autonomous rendezvous and proximity operation control algorithms.
    % The UKF and ground based observation campaign was critical for accurate navigation and allowed for more aggressive maneuvers. 
        Performed astronomical research to process  astronomical measurements in order to provide accurate tracking of orbiting spacecraft.
    \item Designed custom astronomical force model for the spacecraft flight experiment.
        Responsible for documenting and explaining technical details in oral and written forms to diverse team.
    % Incorporated the effects of solar radiation pressure, attitude control actuators and sensors, Earth gravitational models, and thruster uncertainties to predict the expected performance of the spacecraft.
    % Analysis predicted a coupling between rotational and translational motion during angular momentum desaturation burns.
    % A orientation profile was developed which minimized the accumulation of undesired angular momentum and allowed for extended quiescent periods in support of orbit determination activities.
    \item Led \$5M lab development program to develop an in-situ attitude dynamics and control simulator.  
    % Responsible for procurement, design, and integration of the largest spherical air-bearing platform in the world.
    Developed software to interface with inertial measurement units, motion capture systems, and control hardware to allow for hardware in the loop system testing.
    % The facility is currently being used to validate future flight attitude control actuators and algorithms. 
    Provided technical assistance and systems support for laboratory hardware.
    \item Developed a method for space based geolocation via time difference of arrival signals.
        Shared results with several governmental agencies during technical meeting and written reports.
    % Implemented geometric techniques to develop a closed form analytical solution to locate a noncooperative electromagnetic signal based on the time of arrival to a formation of satellites.
    \item Led the development of a satellite relative motion simulator using embedded ground based robotic systems.
    % Designed the software to allow for accurate scaling and simulation of spacecraft motion.
    % Directly implemented a motion capture system on embedded hardware to allow for accurate positioning and control. 
        Implemented astronomical algorithms for the emulation of spacecraft motion.
    \item Managed \$4M DARPA space situational awareness contract by leading diverse team of academia, industry, and government in an effort to develop integrated orbit determination software.
        Provided regular reports of technical research progress in technical meetings and written documents.
\end{itemize}

\item[] \textit{Deputy Space Vehicles Lead, Responsive Space Squadron\\Space Development and Test Directorate (SDTD/SDDR)}
    \hfill {May~2009-Aug~2011}
    
\begin{itemize}
    \item Responsible for development, integration, test, \& launch of ORS-1 (Operationally Responsive Space) satellite.
     ORS-1 was the first operational satellite developed under the ORS office and supports US Central Command Battlespace Awareness.
     % An accelerated development cycle allowed for the integration and launch in under 2 years.
    \item Extensive experience with technical management of diverse contractor/government teams leading to successful ORS-1 launch and orbit operation.
    Managed a team of 5 to ensure the correct integration and testing of the space vehicle.
    % Ensured mission requirements were being met during the integration period. 
    \item Resolved a \$600K satellite flight sensor failure.
    Directly managed the repair analysis team to prevent launch delays and ensure the system capability.
    % Monitored the hardware repair process and verified they were completed correctly.
    \item Firsthand experience monitoring 100+ days of integration and testing of ORS-1.
        Presented oral and written reports on space vehicle status to leadership.
    % Sole space vehicles lead during launch site operations at Wallops Island, VA.
    % Ensured correct procedures and standards leading to on-time launch on 29 June 2011
    \item Assessed and served as on-site government inspector of 200+ satellite test plans.
     Verified technical analysis and testing procedures or all flight hardware of ORS-1. 
     % Test plans were critical in ensuring correct performance of both ground and space hardware and critical to mission success.
\end{itemize}

\end{itemize} % AF work
