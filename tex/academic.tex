% !TEX root = ../kulumani_cv.tex

\subsection*{Academic Experience}
\phantomsection
\addcontentsline{toc}{section}{Academic Experience}

\textbf{George Washington University}, Washington DC
    \begin{itemize}
        \item[] \textit{Adjunct Professor, MAE3145 - Orbital Mechanics} \hfill {Fall 2017, 2018}
            \begin{itemize}
                % \item Completely redesigned coursework to emphasize solutions to real-world astrodynamic problems.
                % \item Developed coursework focused on the fundamentals of scientific software development with Python. TODO Discuss more detail about astrodynamics in Python. Put link to software
                \item Developed new coursework focused on fundamentals of astronomical and astrodynamic dynamics
                \item Educated undergraduate students on fundamentals of scientific software in Python. 
                    Programming assignments focused on numerical methods related to orbital mechanics.
                    

            \end{itemize}
        \item[] \textit{Adjunct Professor, MAE3134 - Linear System Dynamics} \hfill {Spring 2017, 2018}
        \begin{itemize}
            % \item Responsible for developing undergraduate coursework covering the fundamentals of linear systems analysis.
            \item Developed undergraduate coursework covering the fundamentals of linear system analysis.
            \item Taught undergraduate students concepts such as systems modelling, solutions to differential equations, frequency response techniques, and state space methodologies. 
        \end{itemize}
        \item[] \textit{Graduate Researcher} \hfill {Aug~2014-Aug~2018}
        \begin{itemize}
            % \item Grader for MAE6246 - Linear Control Systems. 
            % Responsible for grading graduate student homework assignments relating to linear dynamic analysis, linear control system design, and linear algebra. 
            % \item Graduate teaching assistant for Engineering Drawing and Computer graphics course.
            % Responsible for teaching the fundamentals of sound engineering design and use of technical design software.
            % Developed assignments and examinations to test students ability in applying principles in technical drawing.
            \item Conducted independent research in the Flight Dynamics and Control laboratory in the fields of astrodynamics, low-thrust spacecraft propulsion, asteroid landing, and geometric mechanics.
            Developed new results for continuous low thrust orbital transfers for manuevers around asteroids and the Earth-Moon three body problem.
            \item Applying computational geometric mechanics to develop algorithms which preserve the geometric properties of mechanical systems to obtain accurate numerical results and long term stability.
            \item Designed geometric control algorithms for control of spacecraft operations around asteroids.
                This independent research allows for the accurate mapping of the shape of an asteroid using optical measurements.
        \item Presented independent research at several scientific meetings, including AIAA, AAS Astrodynamics Specialist Conference, and American Control Conference.
        \item Published results of independent research in astrodynamics and geometric control in peer reviewed journals.
        \item Developed specialized software in C++ and Python implementing astronomical algorithms.
            This open source software is available to other researchers and impelments best practices in software design and test methodologies.
        \end{itemize}
        \item[] \textit{Aerospace Research Intern, \href{http://www.applieddefense.com/}{Applied Defense Solutions}} \hfill {Jun~2016}
        \begin{itemize}
            \item Performed independent research for characterizing and tracking space objects. The results of this research were presented at scientific meetings.
            \item Developed innovative techniques for extracting useful information theoretic quantities from particle clouds. 
            Created Python software tools for performing \(k\) nearest-neighbor estimation and Monte Carlo estimation for the analysis and tracking of space objects.
            % \item Investigated the conservation of probabilistic properties of Hamiltonian systems.
            \item Implemented parallel processing software in C++ and Python for use on the \href{https://www.mhpcc.hpc.mil/}{Maui High Performance Computing Center (MHPCC)} to track and categorize space objects.
            Developed software and tutorials for the implemention of Java based software on parallel computing clusters.
        \end{itemize}
        % \item[] \textit{Space Scholar, AFRL Scholar Program} \hfill {Jun~2015-Jul~2015}
        % \begin{itemize}
        %     \item Investigated combined translational and rotational control techniques for spacecraft rendezvous and proximity operations in the presence of constraints.
        %     Developed a geometric nonlinear controller which allows for global attitude tracking.
        %     \item The control system is developed directly on the nonlinear manifold and defined globally without the need of local parameterizations.
        %     Attitude constraints are incorporated directly on the nonlinear manifold through the use of barrier functions and allow for excellent convergence properties in the presence of constraints. 
        % \end{itemize}
    \end{itemize}
    
